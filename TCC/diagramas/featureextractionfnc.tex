\tikzstyle{class} = [rectangle split,rectangle split parts=1,draw ,text badly centered, node distance=1cm,scale=0.7]
\begin{tikzpicture}
[edge from parent fork down,sibling distance=5cm,level distance=4cm,
edge from parent/.style={draw,<-,line width=1.2pt}]
\node[class,rectangle split parts=2](FeatureExtractionFnc)
	{
	\nodepart{one}  \textcolor{blue}{\large{\textit{FeatureExtractionFnc}}}
	 \nodepart{two} \begin{tabular}{cc}
	 					 \textit{\textcolor{purple}{ExtractFeature}}(C3,Cz,C4)\\
					\end{tabular}
	}
	child{ node [class, rectangle split parts=2] (PWelch)
		{
		\nodepart{one} \large{\textcolor{blue}{PWelch}}
		\nodepart{two} \begin{tabular}{cc}
							\textcolor{purple}{ExtractFeature}(C3,Cz,C4)\\
						\end{tabular}
		}
	}
	child{ node [class, rectangle split parts=2] (PSD)
		{
		\nodepart{one} \large{\textcolor{blue}{PSD}}
		\nodepart{two} \begin{tabular}{cc}
		\textcolor{purple}{ExtractFeatures}(C3,Cz,C4)\\
		\end{tabular}
		}
	}
	child{ node [class, rectangle split parts=2] (PMTM)
		{
		\nodepart{one} \large{\textcolor{blue}{PMTM}}
		\nodepart{two} \begin{tabular}{cc}
		\textcolor{purple}{ExtractFeatures}(C3,Cz,C4)\\
		\end{tabular}
		}
	};
\end{tikzpicture}