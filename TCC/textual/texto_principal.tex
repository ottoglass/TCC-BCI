\chapter{INTRODUÇÃO}

Neste documento você irá encontrar algumas informações que irão te auxiliar na montagem do \ac{TCC}.


O \textit{software} utilizado para editar e compilar os arquivos é gratuito e pode ser baixado através do endereço eletrônico: 

\textbf{http://texstudio.sourceforge.net/}

Este \emph{template} foi gerado baseado na distribuição \LaTeX~MikTeX versão 2.9, que deve estar instalada para que possa ser utilizado. O \emph{download} pode ser realizado por meio da ferramenta ``MikTeX Net Installer'', disponível no endereço abaixo (verificar versão adequada para o sistema operacional de seu computador).

\textbf{https://miktex.org/download}

O arquivo ``\textbf{principal.tex}'' contém a configuração geral do documento. O texto da monografia deve ser inserido no arquivo ``\textbf{/textual/texto\_principal.tex}'' (podendo ser dividido em arquivos separados para cada capítulo, se desejado).

Este documento está organizado da seguinte forma:
\begin{itemize}
	\item O Capítulo~\ref*{Cap_estrutura} traz uma visão geral de estrutura que pode ser adotada para a escrita do \ac{TCC};
	\item O Capítulo~\ref{Cap_exemplos} apresenta uma série de exemplos que podem auxiliar o aluno na utilização do editor de texto.
\end{itemize}

