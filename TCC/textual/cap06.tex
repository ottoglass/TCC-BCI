\chapter{Conclusão} \label{lab_conclusao}
\par
Em conclus\~ao, o melhor classificador foi o SVM com um kernel linear. O melhor overlap é de 25\% e, dependendo da aplica\c{c}\~ao, qualquer tamanho de janela pode ser utilizado com um pequeno ganho de performance com uma largura de 2s. 
\par
Por fim, a melhor configura\c{c}\~ao utilizada teve uma performance pr\'oxima aos classificadores utilizados durante a competi\c{c}\~ao, apesar de ser claramente inferior \`a primeira posi\c{c}\~ao. 
\begin{table}[h!]
	\centering
	\caption{Resultados da competi\c{c}\~ao BCI-IV 2008}
	\begin{tabularx}{\textwidth}{c|X|c|c|c|c|c|c|c|c|c|c}		
		\hline\hline
		Pos&contributor&$\kappa$&1&2&3&4&5&6&7&8&9  \\ \hline
		$1^{\underline{o}}$&Z. Y. Chin&\textbf{0.60}&0.40&0.21&0.22&	0.95&0.86&0.61&0.56&0.85&0.74 \\ \hline 
		$2^{\underline{o}}$&H. Gan&\textbf{0.58}&0.42&0.21&0.14&0.94&0.71 &0.62&0.61&0.84&0.78 \\ \hline
		$3^{\underline{o}}$&D. Coyle&\textbf{0.46}&0.19&0.12&0.12 &0.77&0.57&0.49&0.38&0.85&0.61 \\ \hline
		$4^{\underline{o}}$&S. Lodder&\textbf{0.43}&0.23&0.31&0.07&0.91& 	0.24&0.42&0.41&0.74&0.53 \\ \hline
		$5^{\underline{o}}$&J. F. D. Saa&\textbf{0.37}&0.20&0.16&0.16&0.73&0.21&0.19&0.39&0.86&0.44 \\ \hline
		$6^{\underline{o}}$&Y. Ping&\textbf{0.25}&0.02&0.09&0.07&0.43&0.25&0.00&0.14&0.76&0.47\\ \hline\hline
		-&L-SVM PMTM&\textbf{0.51}&0.41&0&0.09&0.92&0.58&0.63&0.4&0.83&0.71\\ \hline
	\end{tabularx}
	\label{Tab:BCI2008}
\end{table}

%\section{Trabalhos Futuros}
%\par Ap\'os a conclusão deste TCC foi desenvolvido nova abordagem, utilizando uma combina\c{c}\~ao entre GAN, CNN e SVM que teve um $\kappa$ de $0.4$ para uma janela de 0.5s. Estes algoritmos ainda estão em estado de desenvolvimento e, por este motivo, n\~ao foram inclu\'idos no escopo do presente TCC.

